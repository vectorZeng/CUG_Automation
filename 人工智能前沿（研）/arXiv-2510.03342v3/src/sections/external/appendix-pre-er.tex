\section{\grlatestER{} is a generalist embodied reasoning model}
\label{appendix-pre-er}

\subsection{Evaluation Details: Generality}
To assess an overall approximation of model embodied reasoning performance, we evaluate \grlatestER{} and other multimodal models on a mix of 15 academic benchmarks.
The aggregated results are reported in \cref{fig:gr1.5_er_generality}, and the individual benchmark performance results are shown in Table \ref{table:er-external} and Table \ref{table:er-generality}.
For text-based VQA evaluation benchmarks, we used Gemini 2.5 Flash to grade response accuracy for both multiple-choice and freeform question formats.

The Gemini 2.5 and GPT-5 models were accessed in between September 1, 2025 and September 20, 2025, using default thinking budgets and without tool use.

\begin{table}[ht]
\centering
\begin{tabularx}{\textwidth}{lccccccc}
\toprule
\textbf{Model} & \makecell[c]{GR-ER \\1.5} & \makecell[c]{GR-ER \\1.5} & \makecell[c]{GR-ER} & \makecell[c]{Gemini 2.5 \\ Pro} & \makecell[c]{Gemini 2.5 \\ Flash} & GPT-5 & GPT-5-mini \\
\midrule
\textbf{Thinking} & Yes & No & No & Yes & Yes & Yes & Yes \\
\midrule
Point-Bench & 71.6 & 73.3 & \textbf{75.7} & 62.7 & 61.7 & 43.6 & 39.5 \\
RefSpatial & 48.5 & 41.8 & \textbf{49.3} & 33.6 & 41.2 & 23.5 & 23.0 \\
RoboSpatial-Pointing & \textbf{31.1} & 25.3 & 30.3 & 8.3 & 7.9 & 19.0 & 12.5 \\
Where2Place & \textbf{59.0} & 48.0 & 41.0 & 37.0 & 48.0 & 37.0 & 33.5 \\
\midrule
Spatial average & \textbf{52.6} & 47.1 & 49.1 & 35.4 & 39.7 & 30.8 & 27.1 \\
\midrule
BLINK & 57.8 & 65.2 & 60.1 & 69.2 & 46.1 & \textbf{71.3} & 66.4 \\
CV-Bench & 84.3 & 83.6 & 83.2 & \textbf{85.9} & 85.5 & \textbf{86.1} & \textbf{85.9} \\
ERQA & 54.8 & 47.0 & 45.3 & 56.0 & 47.5 & \textbf{59.0} & 57.3 \\
EmbSpatial & 78.4 & 73.4 & 56.4 & 78.0 & 76.2 & \textbf{81.5} & 78.8 \\
MindCube & 54.7 & 47.7 & 47.4 & \textbf{59.2} & 55.4 & 58.0 & 55.6 \\
RoboSpatial-VQA & \textbf{79.3} & 57.7 & 66.2 & 71.3 & 73.4 & 69.3 & 70.7 \\
SAT & 76.7 & 62.0 & 64.7 & 74.7 & 73.3 & \textbf{86.7} & 81.3 \\
Cosmos-Reason1 & 72.2 & 68.3 & 62.0 & 73.8 & 72.1 & \textbf{79.4} & 76.3 \\
Min Video Pairs & 72.5 & 67.1 & 59.5 & 72.8 & 69.2 & \textbf{77.0} & 73.0 \\
OpenEQA & 55.0 & 50.5 & 38.3 & 55.7 & 45.3 & \textbf{64.4} & 59.2 \\
VSI-Bench & 45.8 & 39.9 & 34.1 & 51.1 & 45.3 & \textbf{52.9} & 46.2 \\
\midrule
QA average & 66.5 & 60.2 & 56.1 & 68.0 & 62.7 & \textbf{71.4} & 68.2 \\
\midrule
ER Score & \textbf{59.6} & 53.7 & 52.6 & 51.7 & 51.2 & 51.1 & 47.7 \\
Overall Average & \textbf{62.8} & 56.7 & 54.2 & 59.3 & 56.5 & 60.6 & 57.3 \\
\bottomrule
\end{tabularx}
\caption{Model performance on a mix of 15 academic embodied reasoning benchmarks. GPT-5 and GPT-5-mini results obtained via API in September 2025.}
\label{table:er-external}
\end{table}

\begin{table}[ht]
\centering
\begin{tabular}{lccccccc}
\toprule
 \textbf{Model} & GR-ER 1.5 & GR-ER 1.5 & GR-ER & \makecell[c]{Pro 2.5} & \makecell[c]{Flash 2.5} & GPT-5 & GPT-5-mini \\
\midrule
\textbf{Thinking} & Yes & No & No & Yes & Yes & Yes & Yes \\
\midrule
MMMU & 80.7 & 79.3 & 67.0 & 82.0 & 79.7 & 82.0 & 78.0 \\
GPQA & 83.3 & 81.3 & 59.6 & 86.4 & 82.8 & 88.4 & 78.3 \\
Aider Polyglot & 57.3 & 44.4 & 16.0 & 82.2 & 56.7 & 81.3 & 66.7\\
\midrule
Average & 73.8 & 68.3 & 47.5 & 83.5 & 73.1 & 83.9 & 74.3 \\
\bottomrule
\end{tabular}
\caption{Model performance on MMMU, GPQA and Aider Polyglot benchmarks. Results for GPT-5 and GPT-5-mini obtained via API with default thinking settings and no tool use in September 2025.}
\label{table:er-generality}
\end{table}

\subsection{Evaluation Details: Pointing} \label{appendix-pointing}
Table~\ref{table:pointing-breakdown} shows detailed breakdown of the evaluation for complex pointing.
\begin{table}[ht]
\centering
\caption{Model performance on complex pointing benchmarks, broken down by subtask.}
\label{table:pointing-breakdown}
\begin{tabular}{l ccccccc}
\toprule
\textbf{Model} & \makecell[c]{GR-ER \\1.5} & \makecell[c]{GR-ER \\1.5} & \makecell[c]{GR-ER} & \makecell[c]{Gemini 2.5 \\ Pro} & \makecell[c]{Gemini 2.5 \\ Flash} & \makecell[c]{GPT-5} & \makecell[c]{GPT-5-mini} \\
\midrule
\textbf{Thinking} & Yes & No & No & Yes & Yes & Yes & Yes \\
\midrule
\multicolumn{8}{l}{\textit{Standard Pointing}} \\
  Point-Bench-Affordance & 70.9 & 76.5 & \textbf{87.9} & 65.3 & 67.8 & 58.1 & 50.0 \\
  Point-Bench-Counting & 86.8 & 86.8 & \textbf{88.4} & 77.5 & 73.1 & 53.7 & 56.8 \\
  Point-Bench-Reasoning & 61.7 & \textbf{69.0} & 64.8 & 55.4 & 49.4 & 33.0 & 28.3 \\
\midrule
\multicolumn{8}{l}{\textit{Steerable Pointing}} \\
  Point-Bench-Steerable & \textbf{67.8} & 61.8 & 65.8 & 53.4 & 61.3 & 38.0 & 32.0 \\
\midrule
\multicolumn{8}{l}{\textit{Spatial Pointing}} \\
  Point-Bench-Spatial & 71.0 & \textbf{72.6} & 71.9 & 62.7 & 57.2 & 35.4 & 30.3 \\
  RefSpatial & 48.5 & 41.8 & \textbf{49.2} & 33.6 & 41.1 & 23.5 & 23.0 \\
  RoboSpatial & \textbf{31.1} & 25.3 & 30.3 & 8.3 & 7.9 & 19.0 & 12.5 \\
  Where2Place & \textbf{59.0} & 48.0 & 41.0 & 37.0 & 48.0 & 37.0 & 33.5 \\
\midrule
\multicolumn{8}{l}{\textit{Point-to-Count}} \\
  PixMo Count & \textbf{80.0} & 65.0 & 60.0 & 76.0 & 64.0 & 73.0 & 77.0 \\
\midrule
  Average & \textbf{52.6} & 47.1 & 49.1 & 35.4 & 39.7 & 30.8 & 27.1 \\
\bottomrule
\end{tabular}
\end{table}

\subsection{Additional Examples} \label{appendix-er-examples}

\cref{fig:gr1.5_er_sample_thoughts_4} illustrates sampled thoughts from \grshortlatestER{} on embodied reasoning tasks.

\begin{figure*}[h]
    \centering
    \includegraphics[width=\textwidth]{src/assets/ER/thought_samples_4.pdf}
    \caption{Sample thoughts from \grshortlatestER{} performing embodied reasoning tasks.}
    \label{fig:gr1.5_er_sample_thoughts_4}
\end{figure*}

\clearpage

\newpage

\section{\grlatest{}: A Physical Agent}






\subsection{Long-horizon benchmarks}
\label{appendix:long-horizon}
Our long-horizon benchmarks evaluate the combination of the \grshortlatestER{} model with the VLA as an autonomous agent. 
\cref{fig:aloha-long-horizon-rollout} shows visuals of the 4 tasks in the ALOHA long-horizon benchmark. The progress is scored as the sum of points scored along each subtask (Table \ref{tab:aloha_long_horizon}). \cref{fig:omega-long-horizon-rollout} shows visuals of the 4 tasks on the Bi-arm Franka long-horizon benchmark. The
progress is scored as the sum of points scored along each subtask (Table \ref{tab:franka_long_horizon}).

\begin{table}[h!]
\begin{tiny}
\centering
\caption{Progress Scores: ALOHA Robot (Long-horizon Benchmark).}
\label{tab:aloha_long_horizon}
\vspace{1pt} % Added 1pt of space after the caption
% Changed to 4 columns, adjusting width (e.g., to 3.7cm from 5cm)
\begin{tabular}{| p{3.7cm} | p{3.7cm} | p{3.7cm} | p{3.7cm} |}
\toprule

% --- SECTION 1 ---
% Changed multicolumn span from 3 to 4
\multicolumn{4}{c}{\vspace{1pt}\textbf{Benchmark: ALOHA Robot - Long-horizon.}\vspace{1pt}} \\
\hline

% --- Row 1 (Headers) ---
% Added the 4th item to this row
\vspace{1pt}\textbf{Trash Sorting: ``Put the compostables into the green bin, the recyclables into the blue bin, and the waste into the black bin''}.\vspace{1pt} &
\vspace{1pt}\textbf{Desk Organization: ``What is the state of the objects in the table? Return them to their original locations''}.\vspace{1pt} &
\vspace{1pt}\textbf{Packing Suitcase: ``Put the hat and socks into the suitcase then pack the colorful shirt that's on the hanger''}.\vspace{1pt} &
\vspace{1pt}\textbf{Blocks in drawer: ``Open each drawer, and put one block in each drawer''}.\vspace{1pt} \\
\hline
% --- Row 1 (Content) ---
% Content Col 1
\begin{itemize}[leftmargin=1pt,topsep=0pt]
\item[] 0.2 is added per item in the correct bin:
\item[] \begin{itemize}[leftmargin=10pt,topsep=0pt]
    \item red grapes in the green bin;
    \item lettuce leaf in the green bin;
    \item aluminum can in the blue bin;
    \item plastic cup in the blue bin;
    \item energy bar wrapper in the black bin.
\end{itemize}
\end{itemize}
&
% Content Col 2
\begin{itemize}[leftmargin=1pt,topsep=1pt]
\item[] 0.2 is added per item in the correct state:
\item[] \begin{itemize}[leftmargin=10pt,topsep=0pt]
    \item red pen in the pen holder;
    \item blue pen in the pen holder;
    \item green marker in the cork tray;
    \item glasses case closed;
    \item laptop closed.
\end{itemize}
\end{itemize}
&
% Content Col 3
\begin{itemize}[leftmargin=1pt,topsep=0pt]
\item[] 0.25 is added per:
\item[] \begin{itemize}[leftmargin=10pt,topsep=0pt]
    \item white beanie in the suitcase;
    \item blue socks in the suitcase;
    \item shirt taken off the hanger;
    \item shirt in the suitcase.
\end{itemize}
\end{itemize}
&
% Content Col 4 (Moved from Row 2)
\begin{itemize}[leftmargin=1pt,topsep=0pt]
\item[] 0.11 is added per:
\item[] \begin{itemize}[leftmargin=10pt,topsep=0pt]
    \item left drawer was opened;
    \item any block in left drawer;
    \item left drawer closed;
    \item middle drawer was opened;
    \item any block in middle drawer;
    \item middle drawer closed;
    \item right drawer was opened;
    \item any block in right drawer;
    \item right drawer closed.
\end{itemize}
\end{itemize}
\\
\bottomrule
\end{tabular}
\end{tiny}
\end{table}

\begin{figure}[ht!]
    \centering
    \includegraphics[width=\textwidth]{src/assets/agent/long_horizon_eval_rollouts.pdf}
    \caption{ALOHA long-horizon benchmark. 
    \label{fig:aloha-long-horizon-rollout}}
\end{figure}

\begin{table}[t!]
\begin{tiny}
\centering
\caption{Progress Scores: Bi-arm Franka (Long-horizon Benchmark).}
\label{tab:franka_long_horizon}
\vspace{1pt} % Added 1pt of space after the caption
% Changed to 4 columns, adjusting width (e.g., to 3.7cm from 5cm)
\begin{tabular}{| p{3.7cm} | p{3.7cm} | p{3.7cm} | p{3.7cm} |}
\toprule

% --- SECTION 1 ---
% Changed multicolumn span from 3 to 4
\multicolumn{4}{c}{\vspace{1pt}\textbf{Benchmark: Bi-arm Franka - Long-horizon.}\vspace{1pt}} \\
\hline

% --- Row 1 (Headers) ---
% Added the 4th item to this row
\vspace{1pt}\textbf{Swap: "Swap the sardines and the yellow bottle"}.\vspace{1pt} &
\vspace{1pt}\textbf{Top shelf to the table: "Put all the objects from the top right shelf onto the table"}.\vspace{1pt} &
\vspace{1pt}\textbf{Mushroom risotto: "Pack all ingredients for a mushroom risotto into the basket"}.\vspace{1pt} &
\vspace{1pt}\textbf{Vegetarian with nut allergy: "I am vegetarian and allergic to nuts. Can you put all the food I can't eat into the basket"}.\vspace{1pt} \\
\hline
% --- Row 1 (Content) ---
% Content Col 1
\begin{itemize}[leftmargin=1pt,topsep=0pt]
\item[] $0.33$ is added per each subtask:
\item[] \begin{itemize}[leftmargin=10pt,topsep=0pt]
    \item lemon juice is in the correct location;
    \item can of sardines is in the correct location;
    \item no unrelated task done.
\end{itemize}
\end{itemize}
&
% Content Col 2
\begin{itemize}[leftmargin=1pt,topsep=0pt]
\item[] $0.25$ per each subtask:
\item[] \begin{itemize}[leftmargin=10pt,topsep=0pt]
    \item rice is on the table;
    \item corn is on the table;
    \item lemon juice is on table;
    \item no unrelated task done.
\end{itemize}
\end{itemize}
&
% Content Col 3
\begin{itemize}[leftmargin=1pt,topsep=0pt]
\item[] $0.25$ per each subtask:
\item[] \begin{itemize}[leftmargin=10pt,topsep=0pt]
    \item mushrooms are in the basket;
    \item rice is in the basket;
    \item stock cubes are in the basket;
    \item no unrelated task done.
\end{itemize}
\end{itemize}
&
% Content Col 4
\begin{itemize}[leftmargin=1pt,topsep=0pt]
\item[] $0.33$ per each subtask:
\item[] \begin{itemize}[leftmargin=10pt,topsep=0pt]
    \item the can of sardines is into the basket;
    \item the granola is into the basket;
    \item no unrelated task done.
\end{itemize}
\end{itemize}
\\
\bottomrule
\end{tabular}
\end{tiny}
\end{table}

\begin{figure}[ht!]
    \centering
    \includegraphics[width=\textwidth]{src/assets/agent/Omega_lh.pdf}
    \caption{Bi-arm Franka long-horizon benchmark. 
    \label{fig:omega-long-horizon-rollout}}
\end{figure}

\clearpage

\subsection{Success rate}
\label{appendix:additional-exps-agentic}
\cref{fig:agentic_sr} shows the \textit{success rate} for the results in Section \ref{sec:results-agentic}.

\begin{figure}[!h]
\centering
\begin{subfigure}{0.7\linewidth}

    \includegraphics[width=\linewidth]{src/assets/agent/sr_agentic_v3.pdf}
\end{subfigure}
\centering
\begin{subfigure}{0.7\linewidth}

    \includegraphics[width=\linewidth]{src/assets/agent/omega_agentic_sr_no_legend.pdf}
\end{subfigure}
    \caption{Long-horizon evaluations for the \grshortlatest{} Agent on ALOHA (top) and Bi-arm Franka (bottom), consisting of tasks that require advanced real-world understanding, tool use, long-horizon task planning, execution and error recovery to successfully complete the complex long-horizon tasks.}
    \label{fig:agentic_sr}
\end{figure}

\clearpage